\documentclass[a4paper,journal]{IEEEtran}
\usepackage{amsmath,amsfonts}
\usepackage{algorithmic}
\usepackage{array}
\usepackage{cite}
\usepackage[caption=false,font=normalsize,labelfont=sf,textfont=sf]{subfig}
\usepackage{textcomp}
\usepackage{stfloats}
\usepackage{url}
\usepackage{verbatim}
\usepackage[english,spanish,mexico]{babel}
\usepackage{blindtext}
\usepackage[utf8]{inputenc}
\usepackage{graphicx}
\usepackage{balance}
\begin{document}
\title{Bolómetro: Principios de Funcionamiento.\\}
\author{Alexis L. Ayala, Leonardo G. Vazquez\\Escuela Superior de Ingeniería Mecánica y Eléctrica Unidad Zacatenco\\ Detección Óptica\\Profesor: Dr. Eduardo M. Casas\\ aayalal1600@alumno.ipn.mx ,lgutierrezv1701@alumno.ipn.mx}
\markboth{Ingenieria en Fotónica, Archivo No.~1 15 de Abril del 2024}{15 de Abril del 2024}
\maketitle
%Resumen y abstract
\selectlanguage{spanish}
\begin{abstract}
En el presente se revisarán los principios físicos que rigen el funcionamiento de un bolómetro, un dispositivo sensible al calor (radiación infrarroja). Se proporcionará una introducción a las teorías elementales de respuesta del bolómetro, de la radiación térmica y del ruido propio del dispositivo, así como una descripción de las partes que lo constituyen y la tarea a realizar por cada una de ellas. En la introducción se traza un breve recorrido por la evolución del bolómetro moderno, se indica cómo su perfil coincide con la clase más común de detectores térmicos y se describen las aplicaciones para las cuales la solución es óptima.
\\También se analizarán las teorías más rigurosas de respuesta y ruido necesarias para una mejor comprensión del tema. Se hablará sobre la radiación infrarroja en general, los detectores de radiación infrarroja, la estructura de un bolómetro, los tipos de bolómetros y las aplicaciones de interés. 
\\El presente artículo ofrece una visión de la tecnología bolométrica, desde sus fundamentos teóricos hasta su aplicación práctica en diversas condiciones, destacando su importancia en la detección y medición de radiación infrarroja y milimétrica en varios campos científicos y tecnológicos. 
\end{abstract}
%%Palabras clave
\begin{quote}
        \bf
        \small
        \emph{Palabras Clave}—-Principios físicos, calor, radiación infrarroja, teorías elementales, respuesta, radiación térmica, ruido, detectores térmicos, detectores, estructura, aplicaciones, tecnología bolométrica, fundamentos teóricos, detección, medición, milimétrica, campos científicos. 
\end{quote}
\selectlanguage{english}
\begin{abstract}
This paper will review the physical principles governing the operation of a bolometer, a device sensitive to heat (infrared radiation). An introduction to the elementary theories of bolometer response, thermal radiation and the device's own noise will be provided, as well as a description of its constituent parts and the task to be performed by each of them. The introduction briefly traces the evolution of the modern bolometer, shows how its profile matches the most common class of thermal detectors, and describes the applications for which the solution is optimal.
\\The more rigorous theories of response and noise necessary for a better understanding of the subject will also be discussed. Infrared radiation in general, infrared radiation detectors, the structure of a bolometer, types of bolometers and applications of interest will be discussed. 
\\This article provides an overview of bolometer technology, from its theoretical foundations to its practical application in various conditions, highlighting its importance in the detection and measurement of infrared and millimeter radiation in various scientific and technological fields. 
\end{abstract}
%%Keywords
\begin{IEEEkeywords}
Physical principles, heat, infrared radiation, elementary theories, response, thermal radiation, noise, thermal detectors, detectors, structure, applications, bolometric technology, theoretical foundations, detection, measurement, millimeter, scientific fields.
\end{IEEEkeywords}
%%Introducción
\selectlanguage{spanish}
\vspace{0.5cm}
\section{Introducción}
\IEEEPARstart{E}{l} bolómetro es un dispositivo sensible al calor, es decir, a la radiación infrarroja. Su funcionamiento se basa en la variación de la resistencia eléctrica de un material en función de la temperatura. La radiación infrarroja es una forma de radiación electromagnética que se encuentra en el espectro electromagnético entre la luz visible y las microondas. La radiación infrarroja es emitida por todos los objetos a temperatura ambiente y es invisible al ojo humano. Los bolómetros son dispositivos que se utilizan para detectar la radiación infrarroja y convertirla en una señal eléctrica.
%%Principio de Funcionamiento de un Bolometro
\section{Principio de funcionamiento de un Bolómetro}
Hoy en día los detectores infrarrojos se dividen en dos grandes grupos.
\begin{itemize}
  \item El primer grupo se conoce como detectores de efecto cuántico, se basa en la liberación de electrones en estructura en forma de rejilla. 
  \item El segundo grupo denominado detectores térmicos responde al calor emitido por los cuerpos cuando son expuestos a radiación. 
\end{itemize}

\end{document}
